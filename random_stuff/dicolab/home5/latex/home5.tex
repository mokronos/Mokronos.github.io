\documentclass{article}

\setlength{\parindent}{0pt}
\usepackage{graphicx}
\usepackage{hyperref}
\usepackage{amsmath}

\graphicspath{{../images/}}

\begin{document}

\section{H-5.1}


Average:
\begin{align}
    m_a = \frac{1}{M} \sum_{a_m\in A} a_m
\end{align}

Variance:
\begin{align}
   \sigma_a^2 = \frac{1}{M} \sum_{a_m\in A} |a_m|^2 - |m_a|^2
\end{align}

In the case of QAM the mean of the points is 0, as there is always another point symmetric to the origin.

$a_m$ is the amplitude of the $m$-th point.
The variace is increasing with the amount of rings of QAM points (M).

\section{H-5.2}

Signal and noise have zero mean.
???

\section{H-5.3}

randn() generates matrix with random numbers from a normal distribution with mean 0 and variance 1.
Add 1i to get the complex part. sqrt(2) normalizes the values.

\begin{verbatim}
M = 4; % Number of transmit signals
N = 2; % Number of receive signals
H = (randn(N,M) + 1i*randn(N,M)) / sqrt(2);

sigma_n = 0.1; % Noise std
n = (randn(N,1) + 1i*randn(N,1)) * sigma_n;
\end{verbatim}

\section{H-5.4}

Given tx and rx vectors as the sybols transmitted and received, respectively, one can compare them and count how many don't align.
Get length of one vector.
Then devide number of errors by length of the vector of symbols.

\begin{verbatim}
num_errors = sum(tx ~= rx); % Count the number of symbol errors
total_symbols = length(tx); % Total number of transmitted symbols
ser = num_errors / total_symbols; % Calculate the SER
\end{verbatim}

\section{H-5.5}
Rayleight distribution.

pdf:

\begin{equation}
f_p(p) =
    \begin{cases}
        \frac{p}{\sigma^2} e^{-\frac{p^2}{2\sigma^2}} & \text{for } p \geq 0 \\
        0 & \text{for } p < 0
    \end{cases}
\end{equation}

\section{H-5.6}

MRC is trying to combine the signals from the different antennas to get the best signal.
This is done by making each signal proportional to the signal level and inversely proportional to the noise level.

Matrix/vector notation ???

\section{H-5.7}

???

\section{H-5.8}

AS needs less hardware, as it doen't need to combine the signals, but can use the one with the highest signal stregth.

\section{H-5.9}

The amount of signal points to check should just be the amount for one QAM signal times the number of antennas. $C = 2^M N_T$

\end{document}
